\documentclass[
11pt, % The default document font size, options: 10pt, 11pt, 12pt
%codirector, % Uncomment to add a codirector to the title page
]{charter} 




% El títulos de la memoria, se usa en la carátula y se puede usar el cualquier lugar del documento con el comando \ttitle
\titulo{Dispositivo Conversor de Audio Digital-Analógico de Alta Fidelidad} 

% Nombre del posgrado, se usa en la carátula y se puede usar el cualquier lugar del documento con el comando \degreename
\posgrado{Carrera de Especialización en Sistemas Embebidos} 
%\posgrado{Carrera de Especialización en Internet de las Cosas} 
%\posgrado{Carrera de Especialización en Intelegencia Artificial}
%\posgrado{Maestría en Sistemas Embebidos} 
%\posgrado{Maestría en Internet de las cosas}

% Tu nombre, se puede usar el cualquier lugar del documento con el comando \authorname
\autor{Flavio Miravete} 

% El nombre del director y co-director, se puede usar el cualquier lugar del documento con el comando \supname y \cosupname y \pertesupname y \pertecosupname
\director{Dr. Ing. Anibal Zanini}
\pertenenciaDirector{FIUBA} 
% FIXME:NO IMPLEMENTADO EL CODIRECTOR ni su pertenencia
%\codirector{John Doe} % para que aparezca en la portada se debe descomentar la opción 
% codirector en el documentclass
%\pertenenciaCoDirector{FIUBA}

% Nombre del cliente, quien va a aprobar los resultados del proyecto, se puede usar con el comando \clientename y \empclientename
\cliente{CESE}
\empresaCliente{LSE - FIUBA}

% Nombre y pertenencia de los jurados, se pueden usar el cualquier lugar del documento con el comando \jurunoname, \jurdosname y \jurtresname y \perteunoname, \pertedosname y \pertetresname.
\juradoUno{Nombre y Apellido (1)}
\pertenenciaJurUno{pertenencia (1)} 
\juradoDos{Nombre y Apellido (2)}
\pertenenciaJurDos{pertenencia (2)}
\juradoTres{Nombre y Apellido (3)}
\pertenenciaJurTres{pertenencia (3)}
 
\fechaINICIO{22 de agosto de 2023}		%Fecha de inicio de la cursada de GdP \fechaInicioName
\fechaFINALPlan{24 de octubre de 2023} 	%Fecha de final de cursada de GdP
\fechaFINALTrabajo{2 de diciembre de 2024}	%Fecha de defensa pública del trabajo final


\begin{document}

\maketitle
\thispagestyle{empty}
\pagebreak


\thispagestyle{empty}
{\setlength{\parskip}{0pt}
\tableofcontents{}
}
\pagebreak


\section*{Registros de cambios}
\label{sec:registro}


\begin{table}[ht]
\label{tab:registro}
\centering
\begin{tabularx}{\linewidth}{@{}|c|X|c|@{}}
\hline
\rowcolor[HTML]{C0C0C0} 
Revisión & \multicolumn{1}{c|}{\cellcolor[HTML]{C0C0C0}Detalles de los cambios realizados} & Fecha      \\ \hline
0      & Creación del documento                                 & 22/08/2023 \\ \hline
1      & Se completa hasta el punto 5 inclusive                 & 04/09/2023 \\ \hline
2      & Se completa hasta el punto 9 inclusive					& 11/09/2023 \\ \hline
%		  Se puede agregar algo más \newline
%		  En distintas líneas \newline
%		  Así                                                    & dd/mm/aaaa \\ \hline
%3      & Se completa hasta el punto 11 inclusive                & dd/mm/aaaa \\ \hline
%4      & Se completa el plan	                                 & dd/mm/aaaa \\ \hline
\end{tabularx}
\end{table}

\pagebreak



\section*{Acta de constitución del proyecto}
\label{sec:acta}

\begin{flushright}
Buenos Aires, \fechaInicioName
\end{flushright}

\vspace{2cm}

Por medio de la presente se acuerda con el Ing. \authorname\hspace{1px} que su Trabajo Final de la \degreename\hspace{1px} se titulará ``\ttitle'', consistirá esencialmente en el desarrollo y la implementación de un prototipo de un dispositivo capaz de convertir diferentes formatos de audio digital en una señal analógica. Tendrá un presupuesto preliminar estimado de 669 hs de trabajo y \textcolor{black}{USD 41.200}, con fecha de inicio \fechaInicioName\hspace{1px} y fecha de presentación pública \fechaFinalName.

Se adjunta a esta acta la planificación inicial.

\vfill

% Esta parte se construye sola con la información que hayan cargado en el preámbulo del documento y no debe modificarla
\begin{table}[ht]
\centering
\begin{tabular}{ccc}
\begin{tabular}[c]{@{}c@{}}Dr. Ing. Ariel Lutenberg \\ Director posgrado FIUBA\end{tabular} & \hspace{2cm} & \begin{tabular}[c]{@{}c@{}}\clientename \\ \empclientename \end{tabular} \vspace{2.5cm} \\ 
\multicolumn{3}{c}{\begin{tabular}[c]{@{}c@{}} \supname \\ Director del Trabajo Final\end{tabular}} \vspace{2.5cm} \\
%\begin{tabular}[c]{@{}c@{}}\jurunoname \\ Jurado del Trabajo Final\end{tabular}     &  & \begin{tabular}[c]{@{}c@{}}\jurdosname\\ Jurado del Trabajo Final\end{tabular}  \vspace{2.5cm}  \\
%\multicolumn{3}{c}{\begin{tabular}[c]{@{}c@{}} \jurtresname\\ Jurado del Trabajo Final\end{tabular}} \vspace{.5cm}                                                                     
\end{tabular}
\end{table}




\section{1. Descripción técnica-conceptual del proyecto a realizar}
\label{sec:descripcion}

%\begin{consigna}{black}
Cualquier dispositivo que actúe como fuente de sonido digital como un reproductor de CD o Blu-ray, un TV digital, una consola de juegos, un teléfono móvil, un reproductor de música portátil o una computadora necesitará un DAC, ya sea integrado o conectado, para convertir su audio digital a analógico antes de ser amplificado y enviado a un parlante o auricular.

Los datos de audio digital se pueden almacenar en una variedad de frecuencias de muestreo, profundidades de bits, formatos de codificación y compresión, pero sin importar cómo se haga, el trabajo de un DAC es convertir a partir del formato binario la señal de audio en analógica intentando conservarla lo más parecida a la grabación de audio original.
 
En los últimos años han aparecido dispositivos DAC que se insertan en la cadena de audio de cualquier sistema de sonido HiFi en ambientes como producción musical, estudios de grabación, hogar y preferidos por el público audiófilo. 

Por otra parte, hoy existe una amplia variedad de fuentes de audio digital disponibles, por ejemplo:

\begin{itemize}
	\item Archivos con codificación sin pérdida (FLAC, WAV, AIFF, APE).
	\item Archivos con codificación con pérdida (AAC, MP3, WMA, OGG).
	\item Puerto S/PDIF.
	\item Puerto AES-EBU.
	\item Dispositivos Bluetooth (smartphones).
	\item Puertos con capacidad de leer otros formatos como PCM, DSD, MQA, DoP. 
\end{itemize}

La codificación sin pérdidas permite desde los datos digitales decodificados reproducir la información exactamente igual a la original.
La codificacion con pérdidas aprovecha la característica del oido humano que enmascara frecuencias cercanas con distintas amplitudes. Esta información no es percibida por el oido, por lo tanto no se codifica.    

Por las distintas fuentes enumeradas se detecta la necesidad de contar con un único dispositivo que pueda convertir varios de estos formatos en sonido analógico. Esto permitirá reducir el uso de múltiples equipos de interface para cada fuente de audio digital.

El objetivo de este proyecto es diseñar el hardware electrónico de un dispositivo DAC que pueda leer al menos 3 tipos de fuentes de audio digital, incluya un display alfanumérico o gráfico que permita visualizar la operación del dispositivo, cuente con botones de operación según lo que se especifique en su diseño y tenga una salida analógica para la señal de audio.
  
En la Figura \ref{fig:diagBloques} se presenta el diagrama en bloques del sistema. Se observa un esquema lógico de los componentes y funciones mas importantes que podemos encontrar en un DAC. Luego, existen combinaciones entre las interfaces que comunmente se utilizan y los formatos  de audio que se transfieren a través de ellas.

\begin{figure}[htpb]
\centering 
\includegraphics[width=0.95\textwidth]{./Figuras/DAC.png}
\caption{Diagrama en bloques del sistema}
\label{fig:diagBloques}
\end{figure}

\vspace{2cm}
%\end{consigna}


\section{2. Identificación y análisis de los interesados}
\label{sec:interesados}

\begin{table}[ht]
%\caption{Identificación de los interesados}
%\label{tab:interesados}
\begin{tabularx}{\linewidth}{@{}|l|X|X|l|@{}}
\hline
\rowcolor[HTML]{C0C0C0} 
Rol           & Nombre y Apellido & Organización 	& Puesto 	\\ \hline
Cliente       & \supname          &\empclientename	&        	\\ \hline
Responsable   & \authorname       & FIUBA        	& Alumno 	\\ \hline
Orientador    & \supname	      & \pertesupname 	& Director Trabajo final \\ \hline
Usuario final & Público general   & -             	& -       	\\ \hline
\end{tabularx}
\end{table}

\begin{itemize}
	\item Responsable: es también el impulsor del proyecto. Posee conocimientos generales de audio y la viva intencion de fundar una empresa de productos de este rubro. 
	\item Colaboradores: en este momento está abierta la búsqueda de colaboradores que podrán sumarse con el rol de codirector o como parte del equipo.
	\item Orientador: es una persona de muy extensa experiencia en el ámbito académico e industrial, entre varios temas de su dominio, es experto en temas de control automático y en procesamiento de señales digitales.
	\item Usuario final: es una persona entusiasta del audio o audiófilo.
\end{itemize}

\section{3. Propósito del proyecto}
\label{sec:proposito}

\begin{consigna}{black}
El propósito de este proyecto es proveer un producto que pueda insertarse dentro de un sistema de audio HiFi, permitiendo sumar fuentes de audio alternativas a las más populares como el reproductor de discos o la bandeja giradiscos.
 
Otro propósito es generar una base de conocimiento en tecnologías de procesamiento de audio digital y el desarrollo de procesos, procedimientos y experiencia local en la industria del audio. 
\end{consigna}





\section{4. Alcance del proyecto}
\label{sec:alcance}

\begin{consigna}{black}
Se define el siguiente alcance para el proyecto:
\begin{itemize}
	\item[•]Diseño y fabricación de un prototipo del hardware electrónico del producto.  
	\item[•]Inclusión de una interface USB para la lectura o recepción de audio digital. 
	\item[•]Inclusión de una interface serie para la recepción de audio digital por el protocolo SPDIF.
	\item[•]Inclusión de una interface Bluetooth para la recepción de audio.
	\item[•]inclusión de un display y teclas de operación.
	\item[•]Desarrollo del firmware para la operación y manejo de un formato digital por interface.
	\item[•]Documentación de los procesos de desarrollo y software.
	\item[•]Diseño y desarrollo de las pruebas.  
\end{itemize}	
Quedan fuera del alcance los siguientes items:
\begin{itemize}
	\item[•]Desarrollo de firmware para el manejo de formatos de audio digital con pérdidas (codificación/decodificación perceptiva).  
	\item[•]Diseños mecánicos del dispositivo (gabinetes o envolturas). 	
\end{itemize}
\end{consigna}


\section{5. Supuestos del proyecto}
\label{sec:supuestos}

\begin{consigna}{black}
Para el desarrollo del presente proyecto se supone que:

\begin{itemize}
	\item[•]Se cuenta con la información teórica mínima y necesaria referida a los protocolos y formatos de audio digital. 
	\item[•]Se tiene acceso a instrumentos y herramientas para el desarrollo del producto (kit de desarrollos, osciloscopios, analizadores, etc.).
	\item[•]Se cuenta con un mínimo de dedicación de 10 horas semanales para el proyecto.
	\item[•]Se tiene disponibilidad de todos los componentes electrónicos que resulten del diseño.
	\item[•]Se realizarán las compras en proveedores locales siempre que sea posible.  
\end{itemize}

\end{consigna}

\section{6. Requerimientos}
\label{sec:requerimientos}

\begin{consigna}{black}
Los requerimientos a considerar para el desarrollo del proyecto son los siguientes:

\begin{enumerate}
	\item Requerimientos de interfaces digitales.
		\begin{enumerate}
			\item El dispositivo debe contar con una entrada digital USB 2.0 tipo B. 
			\item El dispositivo debe contar con una entrada digital S/PDIF con conexión coaxil. Como alternativa deseable puede agregarse a la conexión coaxil una conexión óptica (TOSLINK).
			\item El dispositivo debe contar con una conexión Bluetooth con capacidad para recibir un stream de audio.   
		\end{enumerate}
	\item Requerimientos de los formatos de audio.
		\begin{enumerate}
			\item El dispositivo debe procesar un stream de audio en formato PCM (Pulse Code Modulation) a 44,1/48/88,2/96 KHz.
			\item El dispositivo debe procesar un stream de audio en formato DSD (Direct Stream Digital) a 2,8/3,1/5,6 MHz.
			\item El dispositivo debe procesar al menos los siguientes formatos sobre Bluetooth: SBC, AAC, aptX.  
		\end{enumerate}
	\item Requerimientos de operación.
		\begin{enumerate}		
			\item El dispositivo debe permitir la selección entre las 3 interfaces de audio disponibles por medio de un pulsador. 	
			\item La interface seleccionada, el formato que se esté reproduciendo y sus características deben ser indicados a través de un display alfanumérico.
			\item El dispositivo debe contar con una perilla rotativa para control de volumen. 
			\item El display alfanumérico debe indicar el estado y progreso del proceso de apareamiento de un dispositivo Bluetooth.
		\end{enumerate}
	\item Requerimientos de alimentación y otros.
		\begin{enumerate}
			\item El dispositivo debe contar con una entrada para alimentación de 5 VDC.
			\item El dispositivo debe contar con una salida analógica stereo, no balanceada y de amplitud máxima 2 Vpp.
		\end{enumerate}
	\item Requerimientos del prototipo.
		\begin{enumerate}
			\item El prototipo debe ser modular (es aceptable un módulo por interface).
			\item Los módulos pueden ser circuitos impresos de diseño propio o comerciales.
		\end{enumerate}
	\item Requerimientos de documentación y registro.
		\begin{enumerate}
			\item Todos los desarrollos de hardware y firmware deben estar guardados en un repositorio y bajo control de versión.
			\item Debe desarrollarse una memoria técnica del prototipo.
			\item Debe desarrollarse un informe de avance. 
			\item Deben documentarse todas las pruebas tecnicas diseñadas y sus conclusiones. 
		\end{enumerate}	
\end{enumerate}

\end{consigna}

\section{7. Historias de usuarios (\textit{Product backlog})}
\label{sec:backlog}

\begin{consigna}{black}
A continuación se enumeran las historias de usuarios identificadas. Para poder medir el tamaño de cada historia se ha realizado una ponderación numérica con base en la serie de Fibonacci 0, 1, 2, 3, 5, 8, 13, 21, 34, 55, .... 
Los criterios y las ponderaciones consideradas son las siguientes:
\begin{enumerate}
	\item Dificultad del trabajo a realizar.
	\begin{itemize}
		\item BAJO = 1
		\item MEDIO = 5
		\item ALTO = 13
	\end{itemize}	
	\item Complejidad del trabajo a realizar.
	\begin{itemize}
		\item BAJO = 2
		\item MEDIO = 5
		\item ALTO = 13
	\end{itemize}	
	\item Riesgo del trabajo a realizar.
	\begin{itemize}
		\item BAJO = 1
		\item MEDIO = 3
		\item ALTO = 8
	\end{itemize}	
\end{enumerate}
Luego, cada historia de usuario obtiene un puntaje (Story Points) que resulta de aproximar la suma de su dificultad, su complejidad y su riesgo al número mas cercano de la serie de Fibonacci.

Historias de usuarios:
\begin{itemize}
	\item[•] Como amante de la música en alta calidad deseo conectar mi computadora con mi sistema de audio HiFi para escuchar mis canciones favoritas que guardo en alta resolución. 
	\begin{itemize}
	\item Dificultad: 5
	\item Complejidad: 13
	\item Riesgo: 3
	\item Suma: 21 -- \textbf{Story Points: 21}
	\end{itemize}
	\item[•] Como consumidor de plataformas de música por streaming quiero reproducir mis caciones desde el teléfono para escucharlas en mi sistema de audio HiFi.
	\begin{itemize}
	\item Dificultad: 13
	\item Complejidad: 13
	\item Riesgo: 3
	\item Suma: 29 -- \textbf{Story Points: 34}
	\end{itemize}
	\item[•] Como entusiasta de las películas y los recitales deseo conectar el audio de mi televisor a mi sistema de sonido HiFi para experimentar el ambiente de una sala de cine.
	\begin{itemize}
	\item Dificultad: 5
	\item Complejidad: 5
	\item Riesgo: 3
	\item Suma: 13 -- \textbf{Story Points: 13}
	\end{itemize}
	\item[•] Como audiófilo, al reproducir audio digital me interesa conocer la profundidad de bits y la frecuencia de muestreo de la música que estoy escuchando para poder comparar y entrenar mi percepción musical.
	\begin{itemize}
	\item Dificultad: 1
	\item Complejidad: 2
	\item Riesgo: 1
	\item Suma: 4 -- \textbf{Story Points: 5}
	\end{itemize}
	\item[•] Como sonidista profesional quiero escuchar las mezclas y ediciones que elaboro en mi estudio para evaluar y corregir mi producción musical.
	\begin{itemize}
	\item Dificultad: 8
	\item Complejidad: 13
	\item Riesgo: 8
	\item Suma: 29 -- \textbf{Story Points: 34}
	\end{itemize}   
\end{itemize}
%Descripción: En esta sección se deben incluir las historias de usuarios y su ponderación (\textit{history points}). Recordar que las historias de usuarios son descripciones cortas y simples de una característica contada desde la perspectiva de la persona que desea la nueva capacidad, generalmente un usuario o cliente del sistema. La ponderación es un número entero que representa el tamaño de la historia comparada con otras historias de similar tipo.

%El formato propuesto es: "como [rol] quiero [tal cosa] para [tal otra cosa]."

%Se debe indicar explícitamente el criterio para calcular los \textit{story points} de cada historia
\end{consigna}

\section{8. Entregables principales del proyecto}
\label{sec:entregables}

\begin{consigna}{black}
Los entregables del proyecto son:

\begin{itemize}
	\item Manual de uso.
	\item Diagrama de circuitos esquemáticos.
	\item Código fuente del firmware.
	\item Prototipo funcional (hardware).
	\item Informe de avance.
	\item Informe final.
\end{itemize}

\end{consigna}

\section{9. Desglose del trabajo en tareas}
\label{sec:wbs}

\begin{consigna}{black}
Se observa a continuación el desglose de tareas del proyecto (WBS, Work Breakdown Struture):

\begin{enumerate}
\item Gestión y documentación del proyecto (102 hs).
	\begin{enumerate}
	\item Realización del plan de proyecto (10 hs).
	\item Control de avance (6 hs).
	\item Documentación del informe de avance (10 hs).
	\item Documentación de la memoria técnica (60 hs).
	\item Documentación de las pruebas (16 hs).
	\end{enumerate}
\item Búsqueda de información y adquisiciones (68 hs).
	\begin{enumerate}
	\item Recopilación de información de mercado (10 hs).
	\item Recopilación de información teórica, normas, documentos y librerías de las interfaces digitales (20 hs).
	\item Recopilación de información teórica, normas, documentos y librerías de los formatos de audio considerados (20 hs).
	\item Búsqueda y selección de los componentes hardware, módulos y/o placas de evaluación (12 hs).
	\item Adquisición del hardware seleccionado (6 hs).
	\end{enumerate}
\item Desarrollo del prototipo (199 hs). 
	\begin{enumerate}
	\item Diseño del circuito esquemático para interfaces USB y S/PDIF (20 hs).
	\item Diseño del circuito esquemático para interface Bluetooth (30 hs).
	\item Diseño del circuito esquemático para display, pulsadores y otros (15 hs).
	\item Validaciones y simulaciones (20 hs). 
	\item Diseño del PCB de las interfaces (40 hs).
	\item Diseño del PCB para el display y otros componentes (20 hs).
	\item Compra de componentes y módulos (4 hs).
	\item Fabricación de PCBs (20 hs).
	\item Armado de PCBs (10 hs).
	\item Desarrollo de firmware de prueba (10 hs).
	\item Ejecución de pruebas (10 hs).
	\end{enumerate}
\item Desarrollo del firmware (230 hs).
	\begin{enumerate}
	\item Diseño de la estructura del programa, módulos y máquinas de estado generales (40 hs).
	\item Desarrollo detallado del firmware para interfaces USB y S/PDIF (40 hs).
	\item Desarrollo detallado del firmware para interfaces Bluetooth (40 hs).
	\item Desarrollo detallado del firmware para display y operación (40 hs).
	\item Diseño y desarrollo detallado de módulos y funciones para manejo de formatos de audio (40 hs).
	\item Diseño y desarrollo detallado de funciones auxiliares y de depuración (30 hs)
	\end{enumerate}
\item Verificación y validación (70 hs).	
	\begin{enumerate}
	\item Diseño y ejecución de pruebas unitarias (30 hs).
	\item Diseño y ejecución de pruebas integrales (40 hs).
	\end{enumerate}	
\end{enumerate}

Cantidad total de horas: (669 hs).

\end{consigna}

\section{10. Diagrama de Activity On Node}
\label{sec:AoN}

\begin{consigna}{red}
Armar el AoN a partir del WBS definido en la etapa anterior. 

%La figura \ref{fig:AoN} fue elaborada con el paquete latex tikz y pueden consultar la siguiente referencia \textit{online}:

%\url{https://www.overleaf.com/learn/latex/LaTeX_Graphics_using_TikZ:_A_Tutorial_for_Beginners_(Part_3)\%E2\%80\%94Creating_Flowcharts}



\begin{figure}[htpb]
\centering 
\includegraphics[width=.8\textwidth]{./Figuras/AoN.png}
\caption{Diagrama de \textit{Activity on Node}.}
\label{fig:AoN}
\end{figure}

Indicar claramente en qué unidades están expresados los tiempos.
De ser necesario indicar los caminos semicríticos y analizar sus tiempos mediante un cuadro.
Es recomendable usar colores y un cuadro indicativo describiendo qué representa cada color, como se muestra en el siguiente ejemplo:

\end{consigna}

\section{11. Diagrama de Gantt}
\label{sec:gantt}

\begin{consigna}{red}

Existen muchos programas y recursos \textit{online} para hacer diagramas de Gantt, entre los cuales destacamos:

\begin{itemize}
\item Planner
\item GanttProject
\item Trello + \textit{plugins}. En el siguiente link hay un tutorial oficial: \\ \url{https://blog.trello.com/es/diagrama-de-gantt-de-un-proyecto}
\item Creately, herramienta online colaborativa. \\\url{https://creately.com/diagram/example/ieb3p3ml/LaTeX}
\item Se puede hacer en latex con el paquete \textit{pgfgantt}\\ \url{http://ctan.dcc.uchile.cl/graphics/pgf/contrib/pgfgantt/pgfgantt.pdf}
\end{itemize}

Pegar acá una captura de pantalla del diagrama de Gantt, cuidando que la letra sea suficientemente grande como para ser legible. 
Si el diagrama queda demasiado ancho, se puede pegar primero la ``tabla'' del Gantt y luego pegar la parte del diagrama de barras del diagrama de Gantt.

Configurar el software para que en la parte de la tabla muestre los códigos del EDT (WBS).\\
Configurar el software para que al lado de cada barra muestre el nombre de cada tarea.\\
Revisar que la fecha de finalización coincida con lo indicado en el Acta Constitutiva.

En la figura \ref{fig:gantt}, se muestra un ejemplo de diagrama de Gantt realizado con el paquete de \textit{pgfgantt}. En la plantilla pueden ver el código que lo genera y usarlo de base para construir el propio.

\begin{figure}[htbp]
\begin{center}
\begin{ganttchart}{1}{12}
  \gantttitle{2020}{12} \\
  \gantttitlelist{1,...,12}{1} \\
  \ganttgroup{Group 1}{1}{7} \\
  \ganttbar{Task 1}{1}{2} \\
  \ganttlinkedbar{Task 2}{3}{7} \ganttnewline
  \ganttmilestone{Milestone o hito}{7} \ganttnewline
  \ganttbar{Final Task}{8}{12}
  \ganttlink{elem2}{elem3}
  \ganttlink{elem3}{elem4}
\end{ganttchart}
\end{center}
\caption{Diagrama de Gantt de ejemplo}
\label{fig:gantt}
\end{figure}


\begin{landscape}
\begin{figure}[htpb]
\centering 
\includegraphics[height=.85\textheight]{./Figuras/Gantt-2.png}
\caption{Ejemplo de diagrama de Gantt rotado}
\label{fig:diagGantt}
\end{figure}

\end{landscape}

\end{consigna}


\section{12. Presupuesto detallado del proyecto}
\label{sec:presupuesto}

\begin{consigna}{red}
Si el proyecto es complejo entonces separarlo en partes:
\begin{itemize}
	\item Un total global, indicando el subtotal acumulado por cada una de las áreas.
	\item El desglose detallado del subtotal de cada una de las áreas.
\end{itemize}

IMPORTANTE: No olvidarse de considerar los COSTOS INDIRECTOS.

\end{consigna}

\begin{table}[htpb]
\centering
\begin{tabularx}{\linewidth}{@{}|X|c|r|r|@{}}
\hline
\rowcolor[HTML]{C0C0C0} 
\multicolumn{4}{|c|}{\cellcolor[HTML]{C0C0C0}COSTOS DIRECTOS} \\ \hline
\rowcolor[HTML]{C0C0C0} 
Descripción &
  \multicolumn{1}{c|}{\cellcolor[HTML]{C0C0C0}Cantidad} &
  \multicolumn{1}{c|}{\cellcolor[HTML]{C0C0C0}Valor unitario} &
  \multicolumn{1}{c|}{\cellcolor[HTML]{C0C0C0}Valor total} \\ \hline
 &
  \multicolumn{1}{c|}{} &
  \multicolumn{1}{c|}{} &
  \multicolumn{1}{c|}{} \\ \hline
 &
  \multicolumn{1}{c|}{} &
  \multicolumn{1}{c|}{} &
  \multicolumn{1}{c|}{} \\ \hline
\multicolumn{1}{|l|}{} &
   &
   &
   \\ \hline
\multicolumn{1}{|l|}{} &
   &
   &
   \\ \hline
\multicolumn{3}{|c|}{SUBTOTAL} &
  \multicolumn{1}{c|}{} \\ \hline
\rowcolor[HTML]{C0C0C0} 
\multicolumn{4}{|c|}{\cellcolor[HTML]{C0C0C0}COSTOS INDIRECTOS} \\ \hline
\rowcolor[HTML]{C0C0C0} 
Descripción &
  \multicolumn{1}{c|}{\cellcolor[HTML]{C0C0C0}Cantidad} &
  \multicolumn{1}{c|}{\cellcolor[HTML]{C0C0C0}Valor unitario} &
  \multicolumn{1}{c|}{\cellcolor[HTML]{C0C0C0}Valor total} \\ \hline
\multicolumn{1}{|l|}{} &
   &
   &
   \\ \hline
\multicolumn{1}{|l|}{} &
   &
   &
   \\ \hline
\multicolumn{1}{|l|}{} &
   &
   &
   \\ \hline
\multicolumn{3}{|c|}{SUBTOTAL} &
  \multicolumn{1}{c|}{} \\ \hline
\rowcolor[HTML]{C0C0C0}
\multicolumn{3}{|c|}{TOTAL} &
   \\ \hline
\end{tabularx}%
\end{table}


\section{13. Gestión de riesgos}
\label{sec:riesgos}

\begin{consigna}{red}
a) Identificación de los riesgos (al menos cinco) y estimación de sus consecuencias:
 
Riesgo 1: detallar el riesgo (riesgo es algo que si ocurre altera los planes previstos de forma negativa)
\begin{itemize}
	\item Severidad (S): mientras más severo, más alto es el número (usar números del 1 al 10).\\
	Justificar el motivo por el cual se asigna determinado número de severidad (S).
	\item Probabilidad de ocurrencia (O): mientras más probable, más alto es el número (usar del 1 al 10).\\
	Justificar el motivo por el cual se asigna determinado número de (O). 
\end{itemize}   

Riesgo 2:
\begin{itemize}
	\item Severidad (S): 
	\item Ocurrencia (O):
\end{itemize}

Riesgo 3:
\begin{itemize}
	\item Severidad (S): 
	\item Ocurrencia (O):
\end{itemize}


b) Tabla de gestión de riesgos:      (El RPN se calcula como RPN=SxO)

\begin{table}[htpb]
\centering
\begin{tabularx}{\linewidth}{@{}|X|c|c|c|c|c|c|@{}}
\hline
\rowcolor[HTML]{C0C0C0} 
Riesgo & S & O & RPN & S* & O* & RPN* \\ \hline
       &   &   &     &    &    &      \\ \hline
       &   &   &     &    &    &      \\ \hline
       &   &   &     &    &    &      \\ \hline
       &   &   &     &    &    &      \\ \hline
       &   &   &     &    &    &      \\ \hline
\end{tabularx}%
\end{table}

Criterio adoptado: 
Se tomarán medidas de mitigación en los riesgos cuyos números de RPN sean mayores a...

Nota: los valores marcados con (*) en la tabla corresponden luego de haber aplicado la mitigación.

c) Plan de mitigación de los riesgos que originalmente excedían el RPN máximo establecido:
 
Riesgo 1: plan de mitigación (si por el RPN fuera necesario elaborar un plan de mitigación).
  Nueva asignación de S y O, con su respectiva justificación:
  - Severidad (S): mientras más severo, más alto es el número (usar números del 1 al 10).
          Justificar el motivo por el cual se asigna determinado número de severidad (S).
  - Probabilidad de ocurrencia (O): mientras más probable, más alto es el número (usar del 1 al 10).
          Justificar el motivo por el cual se asigna determinado número de (O).

Riesgo 2: plan de mitigación (si por el RPN fuera necesario elaborar un plan de mitigación).
 
Riesgo 3: plan de mitigación (si por el RPN fuera necesario elaborar un plan de mitigación).

\end{consigna}


\section{14. Gestión de la calidad}
\label{sec:calidad}

\begin{consigna}{red}
Elija al menos diez requerientos que a su criterio sean los más importantes/críticos/que aportan más valor y para cada uno de ellos indique las acciones de verificación y validación que permitan asegurar su cumplimiento.

\begin{itemize} 
\item Req \#1: copiar acá el requerimiento.

\begin{itemize}
	\item Verificación para confirmar si se cumplió con lo requerido antes de mostrar el sistema al cliente. Detallar 
	\item Validación con el cliente para confirmar que está de acuerdo en que se cumplió con lo requerido. Detallar  
\end{itemize}

\end{itemize}

Tener en cuenta que en este contexto se pueden mencionar simulaciones, cálculos, revisión de hojas de datos, consulta con expertos, mediciones, etc.  Las acciones de verificación suelen considerar al entregable como ``caja blanca'', es decir se conoce en profundidad su funcionamiento interno.  En cambio, las acciones de validación suelen considerar al entregable como ``caja negra'', es decir, que no se conocen los detalles de su funcionamiento interno.

\end{consigna}

\section{15. Procesos de cierre}    
\label{sec:cierre}

\begin{consigna}{red}
Establecer las pautas de trabajo para realizar una reunión final de evaluación del proyecto, tal que contemple las siguientes actividades:

\begin{itemize}
	\item Pautas de trabajo que se seguirán para analizar si se respetó el Plan de Proyecto original:
	 - Indicar quién se ocupará de hacer esto y cuál será el procedimiento a aplicar. 
	\item Identificación de las técnicas y procedimientos útiles e inútiles que se emplearon, y los problemas que surgieron y cómo se solucionaron:
	 - Indicar quién se ocupará de hacer esto y cuál será el procedimiento para dejar registro.
	\item Indicar quién organizará el acto de agradecimiento a todos los interesados, y en especial al equipo de trabajo y colaboradores:
	  - Indicar esto y quién financiará los gastos correspondientes.
\end{itemize}

\end{consigna}


\end{document}
